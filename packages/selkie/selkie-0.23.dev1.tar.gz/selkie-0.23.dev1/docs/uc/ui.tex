
\section{The UC UI}

This section documents the module {\tt seal.uc.ui}.  The examples
assume that one has done:
\begin{python}
>>> from seal.uc.ui import *
>>> from seal.wsgi import run
>>> from seal.uc.corpus import copy_corpus, Corpus, delete_corpus
>>> from seal.io import ex
\end{python}

\subsection{Introduction}

This is the current implementation of {\tt cld}, which is described in
{\it The Design of CLD.\/}  The module {\tt seal.uc.ui} contains a web
application.  It draws on several previous chapters, especially:
\begin{itemize}
\item The tabular-file database (Chapter~\ref{chap2})
\item The Python interface to the UC data (Chapter~\ref{chap3})
\item The Seal user interface (Chapter~\ref{chap4})
\end{itemize}


\subsection{Item store editor}

Access at a low level is provided by the class {\tt ItemStoreEditor}.
To run it:
\begin{myverb}
>>> copy_corpus(ex.uc, '/tmp/foo')
>>> corpus = Corpus(ex.uc)
>>> run(ItemStoreEditor(corpus))
\end{myverb}
