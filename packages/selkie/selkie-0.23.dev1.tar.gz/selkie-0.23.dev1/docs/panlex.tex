
\section{Panlex}

\subsection{Zip}

One can examine the contents of the original zip file using the
{\tt zip} command.  There are four subcommands:
\begin{mydesc}
\ditem{list} List the filenames.
\ditem{head $f$} Print the first 50 records of file $f$.
\ditem{cat $f$} Print all the records of file $f$.
\ditem{table $f$} The table is like the contents, except that, if
there is a field labeled {\tt ex}, two new columns are added: {\tt ex.tt}
and {\tt ex.lv}.  The former contains the string contents of the
expression and the latter is the language-variety code for the
expression.  One may optionally provide an attribute $a$ and value $v$ to
restrict the listing to records that have value $v$ for attribute $a$.
Nota bene: this command is generally {\it much\/} slower than {\tt cat}.
\end{mydesc}

\subsection{Variety}

A language is a set of varieties.
\begin{myverb}
$ panlex variety deu
lv | lc | vc | sy | am | ex | ex.tt | ex.lv
157 | deu | 0 | t | t | 274 | Deutsch | 157
1349 | deu | 1 | t | t | 18586881 | Masematte | 1349
1845 | deu | 2 | t | t | 18586883 | Hessisch | 1845
9097 | deu | 3 | t | t | 12660638 | doitS | 9097
\end{myverb}
These are all the language varieties corresponding to ISO code
``deu.''  Language variety 157 is deu0, variety 1349 is deu1, and so
on.  I don't know what ``sy'' and ``am'' are.  The name of the variety
is given in the variety itself.  Specifically, an expression (ex) is
the pairing of a string (ex.tt) with an indiciation of which variety it is
written in (ex.lv).

To give another example, Ojibwe (oji) is a macrolanguage comprising
Severn Ojibwa (ojs), Eastern Ojibwa (ojg), Central Ojibwa (ojc),
Northwestern Ojibwa (ojb), Western Ojibwa (ojw), Chippewa (ciw),
Ottawa (otw), and Algonquin (alq).
\begin{myverb}
$ python -m panlex variety oji ojs ojg ojc ojb ojw ciw otw alq
lv | lc | vc | sy | am | ex | ex.tt | ex.lv
30 | ojb | 0 | t | t | 18592962 | Anishinaabemowin | 30
536 | ciw | 0 | t | t | 18586345 | Anishinaabemowin | 536
934 | otw | 0 | t | t | 18593131 | Daawaamwin | 934
4069 | ojw | 0 | t | t | 18592975 | Nakaw?mowin | 4069
5598 | ojs | 1 | t | t | 7505858 | ????? | 5598
6930 | ojg | 0 | t | t | 18592966 | Nishnaabemwin | 6930
6931 | ojc | 0 | t | t | 18592964 | Ojibwe | 6931
6932 | ojs | 0 | t | t | 18592970 | Anishininiimowin | 6932
6933 | ciw | 1 | t | t | 8150 | Central Minnesota Chippewa | 187
7415 | ciw | 2 | t | t | 17070963 | Minnesota Ojibwe | 187
9170 | alq | 1 | t | t | 241072 | ???????? | 9170
19 | alq | 0 | t | t | 45808 | anicin?bemowin | 19
\end{myverb}
%$
The question marks represent Unicode characters that Latex does not handle.
The information here does not appear to be entirely correct.  Panlex
labels a wordlist that Margaret and Howard produced as documenting
variety 536 (ciw0), which is Chippewa.  I would have thought that they
speak Eastern Ojibwa.

\subsection{Dicts}

For each variety, there is a set of
dictionaries.

\begin{myverb}
$ python -m panlex dicts 30 536 934 4069 5598 6930 6931 6932 6933 7415 9170 19
128 | Freelang Ojibwe-English dictionary | 13741 | eng-ciw-Weshki
153 | Freelang Ojibwe-English dictionary | 1319 | ciw-ojw-ojc-ojs-ojg-otw-mic-pot-eng-Weshki
611 | Astronomia Terminaro | 2474 | mul-Rapley
2409 | Swadesh Lists | 207 | art-mul-SL
2815 | Anishinaabemowin–English | 131 | ciw-eng-Noori
2830 | Ezhi-Giigidaang, How We Say It (Pronunciation) | 0 | ciw-eng-Kimewon
4091 | Lexique de la langue algonquine | 0 | alq-fra-Cuoq
3778 | Ojibwe Vocabulary Project | 0 | ciw-eng-Manidoons
3779 | Ojibwe-English Wordlist | 0 | ciw-eng-Weshki
4095 | Travels through the Canadas: Vocabulary of the Algonquin Tongue | 0 | alq-eng-Heriot
4144 | The Ojibwe People’s Dictionary | 0 | eng-ciw-OPD
\end{myverb}
%$
A dictionary may document more than one variety.

\subsection{Dict}

To see information about a dictionary:
\begin{myverb}
$ python -m panlex dict 128
ap | lv
128 | 187
128 | 536

id 128
dt 2007-12-11
tt eng-ciw:Weshki
ur http://www.freelang.net/dictionary/ojibwe.php
bn
au Weshki-ayaad; Charles Lippert; Guy T. Gambill
ti Freelang Ojibwe-English dictionary
pb Freelang
yr 2010
uq 5
ui 128
ul TG 122; FreeLang.English_Ojibwe.wb
li co
ip Every author exercises rights with respect to the part of a list that represents that person’s own contribution.
co Guy T. Gambill
ad gambillgt1@yahoo.com
\end{myverb}
%$
The first lines indicate which varieties the dictionary documents.  In
this case, they are 187 (English, eng0) and 536 (Chippewa, ciw0).

\subsection{Bidicts}

To find out which dictionaries document a particular pair of
varieties.
\begin{myverb}
$ python -m panlex bidicts 187 536
128 | Freelang Ojibwe-English dictionary | 13741 | eng-ciw-Weshki
153 | Freelang Ojibwe-English dictionary | 1319 | ciw-ojw-ojc-ojs-ojg-otw-mic-pot-eng-Weshki
611 | Astronomia Terminaro | 2474 | mul-Rapley
2409 | Swadesh Lists | 207 | art-mul-SL
2830 | Ezhi-Giigidaang, How We Say It (Pronunciation) | 0 | ciw-eng-Kimewon
3778 | Ojibwe Vocabulary Project | 0 | ciw-eng-Manidoons
4144 | The Ojibwe People's Dictionary | 0 | eng-ciw-OPD
\end{myverb}
%$
The columns are: dictionary ID ({\tt ap.ap}) title ({\tt ap.ti}),
number of entries (count where {\tt mn.ap==ap}), and short code ({\tt aped.fp}).

\subsection{Bidict}

To extract a bidict:
\begin{myverb}
$ python -m panlex bidict 128 536 187 | uniq > tmp.out
\end{myverb}
%$
The result is ASCII sorted (case sensitive), in two-column format,
with a single tab character as column separator.  Let us think of the
first column as the target language and the second column as the
glossing language.  If a target-language word has multiple glosses,
they produce multiple lines in the file, all sharing the same
target-language word.  (Since the file is sorted, they form a
contiguous block.)  For example, the following occurs in the middle of
{\tt tmp.out}:
\begin{myverb}
aabizh  cut seams open on
aabizhiishin    perk up
aabiziishin     come to
aabiziishin     revive
\end{myverb}
For some reason, the dictionaries sometimes contain duplicate
entries---hence the ``uniq'' in the command line above.


\section{Implementation}

\subsection{Zip file}

\begin{myverb}
f = open_zipfile()
\end{myverb}
The Panlex zip file is \verb|~/src/cl/panlex-20140501-csv.zip|.

Things you can do with a zip file:
\begin{myverb}
f.namelist()      # list of filenames
f.printdir()      # print long listing
s = f.read(name)  # one of the names from namelist
\end{myverb}
The entire file is read as a single string.

The list of Panlex files:
\begin{myverb}
>>> from panlex import open_zipfile
>>> f = open_zipfile()
>>> for nm in f.namelist():
...     print nm
...
panlex-20140501-csv/
panlex-20140501-csv/af.csv
panlex-20140501-csv/mi.csv
panlex-20140501-csv/aped.csv
panlex-20140501-csv/df.csv
panlex-20140501-csv/wc.csv
panlex-20140501-csv/av.csv
panlex-20140501-csv/lv.csv
panlex-20140501-csv/fm.csv
panlex-20140501-csv/ex.csv
panlex-20140501-csv/dm.csv
panlex-20140501-csv/cp.csv
panlex-20140501-csv/md.csv
panlex-20140501-csv/dn.csv
panlex-20140501-csv/cu.csv
panlex-20140501-csv/ap.csv
panlex-20140501-csv/wcex.csv
panlex-20140501-csv/mn.csv
panlex-20140501-csv/apli.csv
\end{myverb}

\subsection{Reading a file}

\subsubsection{Raw contents}

\begin{myverb}
s = raw_contents(fn)
\end{myverb}
The {\tt fn} omits the directory name and the {\tt .csv} suffix.  That
is, legitimate values are ``af,'' ``mi,'' etc.

\subsubsection{Reader}

\begin{myverb}
r = reader(fn)
\end{myverb}
Uses {\tt csv.reader} to parse the csv format.
The return value is an iterator over records, each record being a list
of fields.  The first record contains the field names.
\begin{myverb}
>>> from panlex import reader
>>> r = reader('af')
>>> r.next()
['ap', 'fm']
>>> r.next()
['1636', '24']
\end{myverb}

\subsubsection{Open file}

\begin{myverb}
(hdr, recs) = open_file(fn)
\end{myverb}
The header is the list of field names, and {\tt recs} is an iterator
over the content records.

\subsubsection{Print headers}

Print the database schema: the names and headers of all the files.
\begin{myverb}
>>> from panlex import print_headers
>>> print_headers()
af: ap fm
mi: mn tt
aped: ap q cx im re ed fp etc
df: df mn lv tt
wc: wc dn ex
av: ap lv
lv: lv lc vc sy am ex
fm: fm tt md
ex: ex lv tt td
dm: dm mn ex
cp: lv c0 c1
md: md dn vb vl
dn: dn mn ex
cu: lv c0 c1 loc vb
ap: ap dt tt ur bn au ti pb yr uq ui ul li ip co ad
wcex: ex tt
mn: mn ap
apli: id li pl
\end{myverb}

\subsubsection{Head and cat}

The function {\tt head()} prints the first $n$ records.  The function
{\tt cat()} dumps the contents readably.  \verb|cat(fn,'html')|
produces HTML output.

\subsection{Database tables}

\subsubsection{Where}

Select records containing specified values in a specified field.
The return value is an iterator over records.
\begin{myverb}
>>> from panlex import where
>>> for r in where('lv', 'lc', 'deu'):
...     print '|'.join(r)
...
157|deu|0|t|t|274
1349|deu|1|t|t|18586881
1845|deu|2|t|t|18586883
9097|deu|3|t|t|12660638
\end{myverb}

\subsubsection{Expand expressions}

\begin{myverb}
r = expand_expressions(recs, hdr)
\end{myverb}
Returns an iterator over records.  Two new columns are added: the
first contains the expression's string, and the second contains the
expression's variety.


\subsection{Extracting dictionaries}

\subsubsection{Dict entries}

The function \verb|dict_entry_ids()| returns an iterator over the entry IDs
({\it mnids\/}) for a given dictionary or dictionaries.
\begin{myverb}
>>> from panlex import dict_entries
>>> len(list(dict_entry_ids('128')))
13741
\end{myverb}
The function \verb|dict_entry_table()| returns a table whose keys are
meaning IDs, and whose values are list of pairs of form ({\it lvid, w\/})
where $w$ is a word string.
\begin{myverb}
>>> from panlex import dict_entries
>>> ents = dict_entry_table('128')
>>> len(ents)
13741
>>> mns = list(ents)
>>> mns[0]
'2525999'
>>> ents[mns[0]]
[('187', 'consider'), ('536', 'naagadawaabam')]
>>> ents[mns[1]]
[('187', 'knock against'), ('536', 'bitaakoshkan')]
\end{myverb}

\subsubsection{Bilex pairs}

The function \verb|bilex_pairs()| returns an alphabetically sorted
list of word pairs representing the entries of the given dictionary.
\begin{myverb}
>>> from panlex import bilex_pairs
>>> pairs = bilex_pairs('128','536','187')
>>> pairs[0]
['Aabamadong', 'Fort Hope']
>>> len(pairs)
13739
\end{myverb}
Note that the pair of language IDs is not predictable from the
dictionary.  The dictionary may contain more than two languages, and
even if it only contains two, the dictionary does not specify their
order.


\section{Data schema}

This section lists the data tables.  The following gives the various
data types that occur in the data tables.
\begin{trivlist}\item
\begin{tabular}{|lp{3.75in}|}
\hline
{\it apid} & A dictionary ID.  Panlex calls dictionaries ``approvers.''\\
{\it date} & A date.\\
{\it dnid} & A word-sense ID.  Panlex calls word senses ``denotations.''\\
{\it exid} & An expression ID.  An expression is a word or word sequence in
  a particular language variety.\\
{\it fm} & File format?\\
{\it iso} & An ISO language code.\\
{\it lic} & A 2-letter license code.\\
{\it licid} & A license ID.\\
{\it lvid} & A language-variety ID.\\
{\it mnid} & An entry ID.  Panlex calls entries ``meanings.''\\
{\it num} & A number.\\
{\it str} & A string.\\
{\it t/f} & t or f.\\
{\it uni} & A Unicode code point.\\
{\it url} & A URL.\\
\hline
\end{tabular}
\end{trivlist}

\subsection{Expressions (ex)}

Expressions are used not only for words in dictionaries but also for
parts of speech and dictionary names.
An expression is a word in a particular language variety.  It pairs a
string with a language-variety ID.  Table {\tt ex} lists the expressions:
\begin{trivlist}\item
\begin{tabular}{|llp{3.5in}|}
\hline
\multicolumn{3}{|c|}{\tt ex}\\
\hline
{\tt ex} & {\it exid} & The expression ID.\\
{\tt lv} & {\it lvid} & The language variety of the expression.\\
{\tt tt} & {\it str} & The string.\\
{\tt td} & {\it str} & ``Degraded text'' version.  Contains only lowercase
letters and digits.\\
\hline
\end{tabular}
\end{trivlist}

\subsection{Language varieties (lv)}

Languages are identified by 3-digit ISO codes.  A language variety is
a specialization.  The varieties of a given language are numbered from
0: {\tt eng0}, {\tt eng1}, etc.  There is also a numeric ID for each
language variety.  For example, variety 187 is {\tt eng0}.
\begin{trivlist}\item
\begin{tabular}{|llp{3.5in}|}
\hline
\multicolumn{3}{|c|}{\tt lv}\\
\hline
{\tt lv} & {\it lvid} & The language variety ID.\\
{\tt lc} & {\it iso} & Its ISO language code.\\
{\tt vc} & {\it num} & Language-variety sequence number.  The varieties of a
    particular ISO-coded language are numbered sequentially from 0.\\
{\tt sy} & {\it t/f} & ?\\
{\tt am} & {\it t/f} & ?\\
{\tt ex} & {\it exid} & The name of the variety.  Names are usually given in
    the variety (e.g., the name for German is given as ``Deutsch.''
    But sometimes names are given in English.\\
\hline
\end{tabular}
\end{trivlist}

Additional information about language varieties is given in tables
{\tt cp} and {\tt cu}.  I don't know what these tables contain,
possibly punctuation characters in the language.

\begin{trivlist}\item
\begin{tabular}{|llp{3.5in}|}
\hline
\multicolumn{3}{|c|}{\tt cp}\\
\hline
{\tt lv} & {\it lvid} & A language variety.\\
{\tt c0} & {\it uni} & A code point.\\
{\tt c1} & {\it uni} & A code point.\\
\hline
\end{tabular}
\end{trivlist}

\begin{trivlist}\item
\begin{tabular}{|llp{3.5in}|}
\hline
\multicolumn{3}{|c|}{\tt cu}\\
\hline
{\tt lv} & {\it lvid} & A language variety.\\
{\tt c0} & {\it uni} & A code point.\\
{\tt c1} & {\it uni} & A code point.\\
{\tt loc} & ? & ?\\
{\tt vb} & ? & Values include {\tt pun}, {\tt priv}, {\tt aux}, {\tt
  cit:fin:pri}, {\tt cit:kom:pri}.\\
\hline
\end{tabular}
\end{trivlist}

\subsection{Dictionaries (ap)}

\subsubsection{Dictionaries proper}

An ``approver'' is a dictionary.  A
dictionary documents one or more language varieties.

\begin{trivlist}\item
\begin{tabular}{|llp{3.5in}|}
\hline
\multicolumn{3}{|c|}{\tt av}\\
\hline
{\tt ap} & {\it apid} & The dictionary.\\
{\tt lv} & {\it lvid} & A variety that it documents.\\
\hline
\end{tabular}
\end{trivlist}
Metadata information is contained in the table {\tt ap}.
\begin{trivlist}\item
\begin{tabular}{|llp{3.5in}|}
\hline
\multicolumn{3}{|c|}{\tt ap}\\
\hline
{\tt ap} & {\it apid} & The dictionary ID.\\
{\tt dt} & {\it date} & Registration date.\\
{\tt tt} & {\it str}  & A short identifier, e.g.\ {\tt eng-ciw:Weshki}.\\
{\tt ur} & {\it url}  & The URL.\\
{\tt bn} & {\it str}  & ISBN, perhaps?\\
{\tt au} & {\it str}  & Author.\\
{\tt ti} & {\it str}  & Title.\\
{\tt pb} & {\it str}  & Publisher.\\
{\tt yr} & {\it str}  & Year of publication.\\
{\tt uq} & {\it num}  & Quality?\\
{\tt ui} & {\it apid} & Appears to be the same as {\tt ap}.\\
{\tt ul} & {\it str}  & Some kind of summary line.\\
{\tt li} & {\it lic}  & An IP license code.\\
{\tt ip} & {\it str}  & An IP license statement.\\
{\tt co} & {\it str}  & Company?\\
{\tt ad} & {\it str}  & Email address\\
\hline
\end{tabular}
\end{trivlist}
The table {\tt af} appears to indicate the file format of the original
source for the dictionary.
\begin{trivlist}\item
\begin{tabular}{|llp{3.5in}|}
\hline
\multicolumn{3}{|c|}{\tt af}\\
\hline
{\tt ap} & {\it apid} & The dictionary.\\
{\tt fm} & {\it fm} & The format.  Example values are {\tt html},
    {\tt html-curl}, {\tt pdf-lock/encrypt}, {\tt txt}, {\tt txt-wb},
    {\tt xml}, {\tt pdf-img}, and {\tt db}.
\\
\hline
\end{tabular}
\end{trivlist}
The table {\tt aped} appears to contain Panlex processing information
for dictionaries.
\begin{trivlist}\item
\begin{tabular}{|llp{3.5in}|}
\hline
\multicolumn{3}{|c|}{\tt aped}\\
\hline
{\tt ap} & {\it apid} & The dictionary.\\
{\tt q}  & {\it t/f}  & ?\\
{\tt cx} & {\it num}  & ?\\
{\tt im} & {\it t/f}  & ?\\
{\tt re} & {\it t/f}  & ?\\
{\tt ed} & ?          & ?\\
{\tt fp} & ?          & A code that seems to indicate the documented
    varieties and a one-word abbreviation of the title.  E.g., {\tt eng-ciw-Weshki}.\\
{\tt etc} & {\it str} & Appears to be comments about what work
    needs to be done yet.\\
\hline
\end{tabular}
\end{trivlist}

\subsubsection{Supporting tables}

The {\tt apli} table appears to map 2-letter license codes to
3-letter codes.  I don't know what the codes mean.

\begin{trivlist}\item
\begin{tabular}{|llp{3.5in}|}
\hline
\multicolumn{3}{|c|}{\tt apli}\\
\hline
{\tt id} & licid & License ID\\
{\tt li} & lic & 2-letter code\\
{\tt pl} & ? & 3-letter code\\
\hline
\end{tabular}
\end{trivlist}
The {\tt fm} table contains information about ``fm'' codes, which
appear to be file formats for the original dictionary sources.

\begin{trivlist}\item
\begin{tabular}{|llp{3.5in}|}
\hline
\multicolumn{3}{|c|}{\tt fm}\\
\hline
{\tt fm} & {\it fm} & Format ID?\\
{\tt tt} & {\it str} & Dictionary name??\\
{\tt md} & {\it str} & ?\\
\hline
\end{tabular}
\end{trivlist}

\subsection{Entries (mn)}

A dictionary is a list of entries.  Panlex calls entries ``meanings'' (mn).
\begin{trivlist}\item
\begin{tabular}{|llp{3.5in}|}
\hline
\multicolumn{3}{|c|}{\tt mn}\\
\hline
{\tt mn} & {\it mnid} & The entry ID.\\
{\tt ap} & {\it apid} & The dictionary it belongs to.  The file is
    sorted by this column.\\
\hline
\end{tabular}
\end{trivlist}
The tables {\tt df}, {\tt dm}, and {\tt mi} provide further
information about entries.
The {\tt df} table appears to represent definitions or explanations.
Not all dictionaries have them.
\begin{trivlist}\item
\begin{tabular}{|llp{3.5in}|}
\hline
\multicolumn{3}{|c|}{\tt df}\\
\hline
{\tt df} & ? & Definition ID?\\
{\tt mn} & {\it mnid} & The entry.\\
{\tt lv} & {\it lvid} & Language variety of the definition.\\
{\tt tt} & {\it str} & Definition.\\
\hline
\end{tabular}
\end{trivlist}
The {\tt dm} table possibly represents the semantic domain of an
entry.  Not all dictionaries include it.
\begin{trivlist}\item
\begin{tabular}{|llp{3.5in}|}
\hline
\multicolumn{3}{|c|}{\tt dm}\\
\hline
{\tt dm} & ? & Some kind of ID.\\
{\tt mn} & {\it mnid} & The entry.\\
{\tt ex} & {\it exid} & The semantic domain?\\
\hline
\end{tabular}
\end{trivlist}
I have no idea what {\tt mi} represents.  The values in the {\tt tt}
field are usually IDs of some sort, but occasionally English words.
\begin{trivlist}\item
\begin{tabular}{|llp{3.5in}|}
\hline
\multicolumn{3}{|c|}{\tt mi}\\
\hline
{\tt mn} & ? & ?\\
{\tt tt} & ? & ?\\
\hline
\end{tabular}
\end{trivlist}


\subsection{Word senses (dn)}

\subsubsection{Word senses proper}

An entry is a list of (synonymous) words.
A word-sense is the occurrence of a word within an entry.
Panlex calls word senses ``denotations'' (dn).

\begin{trivlist}\item
\begin{tabular}{|llp{3.5in}|}
\hline
\multicolumn{3}{|c|}{\tt dn}\\
\hline
{\tt dn} & {\it dnid} & The word-sense ID, representing a particular
    word within a particular entry.\\
{\tt mn} & {\it mnid} & The entry.\\
{\tt ex} & {\it exid} & The word.\\
\hline
\end{tabular}
\end{trivlist}
A word-sense may be associated with a part
of speech.
\begin{trivlist}\item
\begin{tabular}{|llp{3.5in}|}
\hline
\multicolumn{3}{|c|}{\tt wc}\\
\hline
{\tt wc} & ? & An ID for the assignment?\\
{\tt dn} & {\it dnid} & The word sense.\\
{\tt ex} & {\tt exid} & The part of speech.\\
\hline
\end{tabular}
\end{trivlist}

The {\tt md} table appears to give the part of speech as written
in the original source.

\begin{trivlist}\item
\begin{tabular}{|llp{3.5in}|}
\hline
\multicolumn{3}{|c|}{\tt md}\\
\hline
{\tt md} & ? & An ID for the assignment?\\
{\tt dn} & {\it dnid} & The word sense.\\
{\tt vb} & {\it str} & The kind of code?\\
{\tt vl} & {\it str} & The original part of speech\\
\hline
\end{tabular}
\end{trivlist}

\subsubsection{Supporting tables}

The {\tt wcex} table is a convenience listing of the part of
speech expressions.

\begin{trivlist}\item
\begin{tabular}{|llp{3.5in}|}
\hline
\multicolumn{3}{|c|}{\tt wcex}\\
\hline
{\tt ex} & {\it exid} & The part-of-speech expression.\\
{\tt tt} & {\it str}  & The part-of-speech string.\\
\hline
\end{tabular}
\end{trivlist}

\end{document}
